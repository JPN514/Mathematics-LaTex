\documentclass{article}
\usepackage[utf8]{inputenc}

\usepackage{amsthm}
\usepackage{amsfonts}
\usepackage{amsmath}
\usepackage{enumitem}
\usepackage{mathtools}
\usepackage{hyperref}
\usepackage[a5paper, total={4in, 5in}]{geometry}

\title{Insta maths (2) 2022.10.28}
\author{jpn514 }
\date{October 2022}

\newtheorem*{theorem}{Theorem}

\begin{document}
\begin{center}
\huge\underline{\textbf{Cauchy Sequences}}\newline
\huge\underline{\textbf{and Convergence}}\newline
\end{center}

\begin{theorem}[Cauchy's Convergence Criterion]
A sequence $(x_{n})_{n=1}^{\infty}$ of real numbers converges if and only if $(x_{n})_{n=1}^{\infty}$ is a Cauchy sequence.
\end{theorem}
\begin{proof}
$(\implies)$\newline
Firstly, let $(x_{n})_{n=1}^{\infty}$ be a sequence of real numbers and suppose it converges to a limit $L\in\mathbb{R}$. That is,
$$\forall\epsilon>0\text{, there exists }N\in\mathbb{N}\text{ such that }
\forall n\geq N\text{, } \vert{x_{n}-L}\vert<\frac{\epsilon}{2}.$$
From this, take $n,m\geq N$ then use the triangle inequality to deduce 
$$\vert{x_{n}-x_{m}}\vert\leq\vert{x_{n}-L}\vert+\vert{x_{m}-L}\vert<\frac{\epsilon}{2}+\frac{\epsilon}{2}=\epsilon\text{.}$$
Thus, $(x_{n})_{n=1}^{\infty}$ is a Cauchy sequence of real numbers.\newline

$(\impliedby)$
Suppose now the converse, that $(x_{n})_{n=1}^{\infty}$ is a Cauchy sequence of real numbers.
We will prove convergence in two steps as follows:
\begin{enumerate}[label=(\roman*)]
    \item $(x_{n})_{n=1}^{\infty}$ is bounded.
    \item $x_n\to L$ as $n\to\infty$.
\end{enumerate}

\begin{proof}[Proof of (i)]
Let $\epsilon=1$. Since $(x_{n})_{n=1}^{\infty}$ is a Cauchy sequence, we have 
\begin{align}
\vert x_{n}-x_{m}\vert<1\text{, } \forall n,m\geq N\text{.}\label{eq:(1)}\tag{1}
\end{align}
Taking $m=N$, from \eqref{eq:(1)} we have,
$$\big\vert \vert x_{n}\vert -\vert x_{N}\vert \big\vert \leq \vert x_{n}-x_{N}\vert <1 \text{, }\forall n\geq N\text{.}$$
Hence, 
\begin{align}
\vert x_{n}\vert < 1 + \vert x_{N}\vert \text{, } \forall n\geq N\text{.}\label{eq:(2)}\tag{2}
\end{align}

Now, let $K=\max\{\vert x_{1} \vert, \vert x_{2} \vert, ..., 1+\vert x_{N} \vert\}$. Then using \eqref{eq:(2)} and $K$ we deduce $\vert x_{n} \vert\leq K\text{, } \forall n\in\mathbb{N}$. Thus, $(x_{n})_{n=1}^{\infty}$ is a bounded sequence of real numbers.
\end{proof}

\begin{proof}[Proof of (ii)]
From the boundedness of $(x_{n})_{n=1}^{\infty}$ and the Bolzano-Weierstrass Theorem, we have existence of a convergent subsequence $(x_{n_{i}})_{i=1}^{\infty}$ where $x_{n_{i}}\to L$ as $i\to\infty$ for some real number $L$. 

Let $\epsilon > 0$. From the convergence of the subsequence, there exists $I_{1}\in\mathbb{N}$ such that, 
\begin{align}
\vert x_{n_{i}}-L\vert<\frac{\epsilon}{2}\text{, }\forall i\geq I_{1}\text{.}\label{eq:(3)}\tag{3}
\end{align}

Furthermore, as $(x_{n})_{n=1}^{\infty}$ is a Cauchy sequence we have,
\begin{align}
\exists N_{1}\in\mathbb{N} \text{ such that } \vert x_{n}-x_{m}\vert< \frac{\epsilon}{2}\text{, } \forall n,m\geq N_{1}\text{.}\label{eq:(4)}\tag{4}
\end{align}

Next, choose $I_{2}\in\mathbb{N}$ such that $I_{2}\geq I_{1}$ and $n_{I_{2}}\geq N_{1}$. 
Set $N_{2}=n_{I_{2}}$. Then when $n\geq N_{2}$, we use \eqref{eq:(3)} and \eqref{eq:(4)} to deduce 
$$\vert x_{n}-L\vert\leq\vert x_{n}-x_{N_{2}}\vert + \vert x_{N_{2}}-L\vert < \frac{\epsilon}{2}+\frac{\epsilon}{2}=\epsilon\text{.}$$
Therefore, $x_n\to L$, as $n\to\infty$ and because $\epsilon >0$ is arbitrary, we have that the sequence  $(x_{n})_{n=1}^{\infty}$ converges to a limit $L\in\mathbb{R}$.
\end{proof}
Having proved both (i) and (ii) we have shown the given Cauchy sequence of real numbers converges to a real number $L$ and this concludes the proof.

\end{proof}

\end{document}
