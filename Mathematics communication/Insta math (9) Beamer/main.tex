% Inbuilt themes in beamer
\documentclass{beamer}

% Theme choice:
\usetheme{Copenhagen}
\usecolortheme{wolverine}
\usepackage{graphicx}
\usepackage{wrapfig}
\usepackage[absolute,overlay]{textpos}
\usepackage{pgfplots}
\graphicspath{ {./images/} }


% Title page details: 
\title{Functional Calculus} 

\author{}
\date{}
\logo{}


\newtheorem*{remark}{Remark}
\newtheorem{prop}{Proposition}[section]
\newtheorem*{notation}{Notation}
\newtheorem*{ex}{Examples:}



\begin{document}

% Title page frame

\begin{frame}{Functional Calculus}

\begin{textblock*}{11cm}(1cm,2cm)
A functional calculus constructs an operator $f(T)$, extending a given function $f$ into a function with an operator argument. For example, this is useful when considering polynomials or the exponential of an operator $T$. Specifically, if $T$ is an $n\times n$ matrix with complex entries, and $f$ is the entire complex valued function given by $f(\zeta)=e^{\zeta}$, we have $$f(T)=e^{T}=I+T+\frac{T^{2}}{2!}+\frac{T^{3}}{3!}+\cdots$$
We can go on to define several functional calculi for different classes of functions together with several types of operators including bounded, sectorial and self-adjoint operators.
\end{textblock*}
\end{frame}

% Remove logo from the next slides
\logo{}


% Outline frame
\begin{frame}{Bounded Holomorphic Functional Calculus}
\begin{textblock*}{11cm}(1cm,2cm)
Let $T$ be a bounded operator on a Banach space $\mathcal{X}$ and let $\Omega\subset\mathbb{C}$ be an open set with the spectrum $\sigma(T)\subset\Omega$. We denote the space of all complex valued holomorphic functions on $\Omega$ by $H(\Omega)$. 
\begin{block}{}
The $H(\Omega)$ functional calculus for bounded operators $\Phi_{B}\colon H(\Omega)\to\mathcal{L}(\mathcal{X})$ for a function $f\in H(\Omega)$, defines the operator $\Phi_{B}(f):=f(T)\in\mathcal{L}(\mathcal{X})$ as follows $$f(T) = \frac{1}{2\pi i}\int_{\gamma}f(\zeta)R_{T}(\zeta)d\zeta = \frac{1}{2\pi i}\int_{\gamma}f(\zeta)(\zeta I - T)^{-1}d\zeta,$$ where the contour $\gamma$ envelopes  $\sigma(T)$ in $\Omega$ and $R_{T}(\zeta)$ is the resolvent of $T$.
\end{block}
\end{textblock*}
\end{frame}

\end{document}