% Inbuilt themes in beamer
\documentclass{beamer}

% Theme choice:
\usetheme{Copenhagen}
\usecolortheme{wolverine}
\usepackage{graphicx}
\usepackage{wrapfig}
\usepackage[absolute,overlay]{textpos}
\usepackage{pgfplots}
\graphicspath{ {./images/} }


% Title page details: 
\title{Heisenberg's Uncertainty Principle} 

\author{}
\date{}
\logo{}


\newtheorem*{remark}{Remark}
\newtheorem{prop}{Proposition}[section]
\newtheorem*{notation}{Notation}
\newtheorem*{ex}{Examples:}



\begin{document}

% Title page frame

\begin{frame}{Fourier Transform and Heisenberg's Inequality}

\begin{textblock*}{11cm}(1cm,1.3cm)
Heisenberg's inequality is an illustration that a function and its Fourier transform cannot be simultaneously "highly localised". This can be deduced intuitively by considering what happens if we were to "spread out" a function f, using some constant $a>0$, we have, $$g(x)=f(ax), \text{ then, } \hat{g}(\xi)=\frac{1}{a}\hat{f}\bigg(\frac{\xi}{a}\bigg).$$
Heisenberg's inequality is a more fundamental statement, specifically for functions in the Schwartz class, denoted $\mathcal{S}(\mathbb{R})$. Informally, a Schwartz class function $f$ is differentiable to all orders and $f$ together with its derivatives are rapidly decreasing as $|x|\to\infty$.
\end{textblock*}
\end{frame}

% Remove logo from the next slides
\logo{}


% Outline frame
\begin{frame}{Fourier Transform and Heisenberg's Inequality}
\begin{textblock*}{11cm}(1cm,1.3cm)

\begin{block}{Heisenberg's Inequality}
Let $f\in\mathcal{S}$ with $$\int_{\mathbb{R}}|f(x)|^{2}dx=1.$$
Then $$\Bigg(\int_{\mathbb{R}}x^{2}|f(x)|^{2}dx\Bigg)\Bigg(\int_{\mathbb{R}}\xi^{2}|\hat{f}(\xi)|^{2}d\xi\Bigg) \geq \frac{\pi}{2}.$$
\end{block}
Here, the two integrals on the left hand side of Heisenberg's inequality measure the spread of the function $f$ and its Fourier transform $\hat{f}$. You may recognise these as the variances of certain continuous random variables with probability distributions $|f(x)|^{2}$ and $|\hat{f}(\xi)|^{2}$, respectively.
\end{textblock*}
\end{frame}

\begin{frame}{Fourier Transform and Heisenberg's Inequality}
\begin{textblock*}{11cm}(1cm,1.3cm)
Since the product of these measures of spread is bounded below by a constant, we deduce that the function $f$ and its Fourier transform cannot be simultaneously 
highly localised/concentrated.


\begin{block}{}
We note that Heisenberg's inequality is equivalent to:
$$\Bigg(\int_{\mathbb{R}}(x-x_{0})^{2}|f(x)|^{2}dx\Bigg)\Bigg(\int_{\mathbb{R}}(\xi-\xi_{0})^{2}|\hat{f}(\xi)|^{2}d\xi\Bigg) \geq \frac{\pi}{2},$$
where $x_{0},\xi_{0}\in\mathbb{R}$; this is a slightly stronger statement.
\end{block}

\begin{remark}
Throughout we have taken the Fourier transform of $f$ to be defined as 
$$\hat{f}(\xi)=\int_{\mathbb{R}}f(x)e^{-ix\xi}dx.$$
\end{remark}

\end{textblock*}
\end{frame}

\end{document}
