% Inbuilt themes in beamer
\documentclass{beamer}

% Theme choice:
\usetheme{Copenhagen}
\usepackage{graphicx}
\usepackage{wrapfig}
\usepackage[absolute,overlay]{textpos}
\graphicspath{ {./images/} }


% Title page details: 
\title{Measure Theory Introduction} 

\author{}
\date{}
\logo{\large \LaTeX{}}


\newtheorem*{remark}{Remark}
\newtheorem{prop}{Proposition}[section]
\newtheorem*{notation}{Notation}
\newtheorem*{ex}{Examples:}



\begin{document}

% Title page frame
\begin{frame}
    \titlepage 
    
    
\begin{textblock*}{10cm}(1cm,4.5cm)
This investigates:
\begin{enumerate}
        \item Outer Measures and Measurable Sets
        \item Borel Sigma Algebra
        \item Measures, Measurable/Measure Spaces
    \end{enumerate}
\end{textblock*}    
\end{frame}

% Remove logo from the next slides
\logo{}


% Outline frame
\begin{frame}{Preliminaries}
\begin{definition}[Outer Measure]
Let $X$ be a set. An outer measure $m^{*}$ on $X$ assigns a non-negative number to each subset of $X$ and satisfies the following:

$(i)$ $m^{*}(\emptyset)=0$; $(ii)$ If $A\subseteq B\subseteq X$ then $m^{*}(A)\leq m^{*}(B)$; \newline
$$(iii) \text{ For } S_{i}\subseteq X \text{ we have, }
m^{*}\big(\bigcup_{i=1}^{\infty}S_{i}\big)\leq \sum_{i=1}^{\infty}m^*(S_{i}).$$ 

\end{definition}

\begin{definition}[Measurable Set]
Let $X$ be a set and suppose $m^{*}$ is an outer measure on $X$. Then $E\subseteq X$ is $m^{*}$-measurable if $\forall T\subseteq X$ we have 
$$m^{*}(T) = m^{*}(T\cap E)+m^{*}(T\setminus E).$$
\end{definition}
\end{frame}


% Lists frame

\begin{frame}{Sigma-Algebras}
\begin{remark}
The outer Lebesgue measure is an outer measure.
\end{remark}
\begin{definition}[Sigma-algebra]
Let $X$ be a set and $\Sigma$ be a collection of subsets of $X$. We say $\Sigma$ is a sigma-algebra on $X$ if:
$(i)$ $\emptyset\in\Sigma$; $(ii)$ If $E\in\Sigma$ then $X\setminus E\in\Sigma$;
$$(iii) \text{ For } E_{n}\in\Sigma \text{ with } n\in\mathbb{N}\text{, } \text{we have, } 
\bigcup_{n=1}^{\infty}E_{n}\in\Sigma.$$
\end{definition}

Let $S\subseteq X$. The sigma-algebra generated by $S$ is $\sigma(S)=\bigcap\big\{\Sigma:\Sigma\supseteq S \text{ and } \Sigma \text{ is a sigma-algebra}\big\}$.


\end{frame}
    
\begin{frame}{Sigma-Algebras}
Consider the case when the set $X$ is a topological space or a metric space. Then we can consider the set of all open subsets of $X$ which we will call $\mathbb{G}$. 
\begin{definition}[Borel sigma-algebra]
The Borel sigma-algebra $\mathfrak{B}$ on $X$ is the sigma-algebra generated by $\mathbb{G}$. We refer to the sets in $\mathfrak{B}=\sigma(\mathbb{G})$ as Borel sets.
\end{definition}

\begin{remark}
Every closed subset of $\mathbb{R}$ is Borel.
\end{remark}
\begin{proof}
Let $T\subseteq\mathbb{R}$ be a closed set. Then $E=\mathbb{R}\setminus T$ is open and $E\in\mathbb{G}$. As $\mathfrak{B}$ contains all open sets we have $E\in\mathfrak{B}$. Since $\mathfrak{B}$ is a sigma-algebra we have $T=\mathbb{R}\setminus E\in\mathfrak{B}$.
\end{proof}

\end{frame} 

\begin{frame}{Measurable Sets and Measures}

\begin{definition}[Lebesgue-measurable Sets]
Let $\lambda^{*}$ denote the outer Lebesgue measure. Then $\lambda^{*}$-measurable sets are called Lebesgue measurable sets. We let $\mathcal{L}$ denote the collection of these sets.
\end{definition}
We note $\mathfrak{B}\subset\mathcal{L}$. That is, Borel sets are Lebesgue measurable. 

\begin{definition}[Measure]
Let $X$ be a set and $\Sigma$ be a sigma-algebra on $X$. The function $\mu:\Sigma\to[0,+\infty)\cup\{\infty\}$ such that 
$(i)\text{ } \mu(\emptyset)=0$; 
$$(ii)\text{ for pairwise disjoint } E_{i}\in\Sigma\text{, }  \mu\big(\bigcup_{i=1}^{\infty}E_{i}\big)=\sum_{i=1}^{\infty}\mu(E_{i}).$$

\end{definition}

\end{frame}

\begin{frame}{Measurable Sets and Measures}
\begin{remark}
\begin{enumerate}
    \item The paramount difference between a measure and an outer measure is the collection of subsets they are defined on. A measure is only defined on subsets in the sigma-algebra on $X$, whereas an outer measure is defined on all subsets $X$.
    \item Note $\emptyset\in\Sigma$ always holds so $m(\emptyset)$ is well-defined.
    \item Property $(ii)$ in the definition holds for both finite and infinite sequences of pairwise disjoint $E_{i}\in\Sigma$.
    \item For $A,B\in\Sigma$ with $A\subseteq B$ then $m(A)\leq m(B)$.
\end{enumerate}
\end{remark}

\begin{example}
Let $X=\mathbb{R}$ and $m^{*}=\lambda^{*}$ be the outer Lebesgue measure. This leads to a measure defined on the collection of $\lambda^{*}$-measurable subsets of $\mathbb{R}$.
\end{example}

\end{frame}

\begin{frame}{Measurable Spaces and Measure Spaces}

\begin{definition}
The pair $(X,\Sigma)$ is called a measurable space.\newline
Let $\mu$ be a measure on $\Sigma$, then the triple $(X,\Sigma,\mu)$ is called a measure space.
\end{definition}
Important examples of measure spaces include probability spaces, which are simply measure spaces with a probability measure. Furthermore, market measures and risk-neutral measures are used in measure spaces for mathematical finance.

\end{frame}



\end{document}
