% Inbuilt themes in beamer
\documentclass{beamer}

% Theme choice:
\usetheme{Copenhagen}

% Title page details: 
\title{Some Homeomorphisms} 
\author{}
\date{}
\logo{\large \LaTeX{}}


\begin{document}

% Title page frame
\begin{frame}
    \titlepage 
\end{frame}

% Remove logo from the next slides
\logo{}


% Outline frame
\begin{frame}{Introduction}
    \begin{definition}
Let $(X,\tau_{X})$ and $(Y,\tau_{Y})$ be topological spaces. Then $f:X\to Y$ is a homeomorphism if $f$ is continuous, bijective and its inverse $f^{-1}:Y\to X$ is continuous.    
    \end{definition}
Throughout we consider the Euclidean topology on $\mathbb{R}$ and the intervals of $\mathbb{R}$, both open and closed.    
\end{frame}


% Lists frame
\section{}
\begin{frame}{}
\begin{block}{}
$1.$ Prove there exists a homeomorphism $f:(0,2)\to(0,18)$.
\end{block}
\newline
\begin{proof}
Remember that a $homeomorphism$ is a bijective continuous function whose inverse is also continuous.
From this we consider the $f$ as follows, 
$$ f: x \mapsto 9x \text{ for } x\in(0,2) $$ which then allows us to define 
$$ f^{-1}:(0,18)\to(0,2) \text{ by } f^{-1}(y)=\frac{y}{9}\text{.} $$
\renewcommand{\qedsymbol}{}
\end{proof}
\end{frame}
    
\begin{frame}
\begin{block}{}
$1.$ Prove there exists a homeomorphism $f:(0,2)\to(0,18)$.
\end{block}
\begin{proof}[Proof cont.]
Now, since $f$,$f^{-1}$ are both polynomials, they are continuous on $\mathbb{R}$ so are clearly continuous on the subset of $\mathbb{R}$ we are working in. To show $f$ is bijective recall that a polynomial $P(x)$ is bijective if and only if $P'(x)$ never changes sign. We can see that $f'(x)=9$, which is constant implying that $f$ is bijective. Therefore, we have that $f$ is a homeomorphism.
\end{proof}
\end{frame} 

\begin{frame}{}
\begin{block}{}
$2.$ Prove there exists a homeomorphism $f:\mathbb{R}\to(-37,37)$.
\end{block}
\newline
\newline Before continuing it is helpful to note that functions with asymptotic properties are effective when considering $\mathbb{R}$ as one of the sets. Below we shall work with some trigonometric functions with the required properties.
\end{frame}

\begin{frame}
\begin{block}{}
$2.$ Prove there exists a homeomorphism $f:\mathbb{R}\to(-37,37)$.
\end{block}
\begin{proof}
We define $f:\mathbb{R}\to(-37,37)$ as follows,
$$f(x)=\frac{74}{\pi}\arctan{(x)}$$ which is a continuous and bijective function. It follows that its inverse $f^{-1}(y)=\tan{\Big(\frac{\pi y}{74}\Big)}$ is continuous also because it is the composition of trigonometric and polynomial functions.
\end{proof}
\end{frame}

\begin{frame}{}
$3.$ Does there exist a homeomorphism $f:\mathbb{R}\to[-1,1]$?
\begin{proof}
Suppose there does exist a homeomorphism $f:\mathbb{R}\to[-1,1]$ and consider its inverse $f^{-1}:[-1,1]\to\mathbb{R}$ which would necessarily be a continuous and bijective function. However, given $f^{-1}$ is continuous and $[-1,1]$ is a closed interval, $f^{-1}$ is bounded and attains a maximum and a minimum by the Extreme Value Theorem and therefore cannot be surjective. Thus, no such homeomorphism exists.
\end{proof}    
\end{frame}



\end{document}