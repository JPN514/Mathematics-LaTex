% Inbuilt themes in beamer
\documentclass{beamer}

% Theme choice:
\usetheme{Copenhagen}
\usepackage{graphicx}
\usepackage{wrapfig}
\usepackage[absolute,overlay]{textpos}
\graphicspath{ {./images/} }

% Title page details: 
\title{Riemann Sphere, Part $1$} 

\author{}
\date{}
\logo{\large \LaTeX{}}


\newtheorem*{remark}{Remark}
\newtheorem{prop}{Proposition}[section]
\newtheorem*{notation}{Notation}
\newtheorem*{ex}{Examples:}


\begin{document}

% Title page frame
\begin{frame}
    \titlepage 
    
    \begin{wrapfigure}{r}{0.25\textwidth}
    \centering
    \includegraphics[width=0.25\textwidth]{images/sphere.jpg}
    
    
\end{wrapfigure}
\begin{textblock*}{10cm}(1cm,4.5cm)
This investigates:
\begin{enumerate}
        \item Extended Complex Numbers
        \item Stereographic Projection
        \item Correspondence 
        \item Visual Representation
    \end{enumerate}
\end{textblock*}    
\end{frame}

% Remove logo from the next slides
\logo{}


% Outline frame
\begin{frame}{Introduction}
The Riemann sphere, named after German mathematician Bernhard Riemann, is a representation of the extended complex numbers. We denote the extended complex numbers as $\mathbb{C}_{\infty}=\mathbb{C}\cup\{\infty\}$, which is simply the complex plane together with a point at $\infty$. Riemann introduced an elegant geometric representation of this by representing all points of $\mathbb{C}$ and the point at $\infty$ on a sphere.
\end{frame}


% Lists frame
\section{}
\begin{frame}{Arithmetic operations in $\mathbb{C}_{\infty}$}
We define the following for any $z\in\mathbb{C}$:
\begin{enumerate}
    \item Addition: $z+\infty=\infty$
    \item Multiplication: $z\times\infty=\infty$
    \item Further operations: for all $z\in\mathbb{C}\setminus\{0\}$:
    $\frac{z}{0}=\infty$, $\frac{z}{\infty}=0$
    \item We define $\frac{0}{\infty}=0$ and $\frac{\infty}{0}=\infty$.
\end{enumerate} 



   Note that $\frac{0}{0}$ and $\frac{\infty}{\infty}$ are not defined.
     $\mathbb{C_{\infty}}$ is not a field due to there being no multiplicative inverse for $\infty$.


\end{frame}
    
\begin{frame}{Stereographic Projection}
We look at the relationship between points on the sphere $\Sigma$ of unit radius and points in the extended complex plane. 
Throughout we consider the complex plane placed horizontally in space, with the plane oriented in such a way that a rotation of $\frac{\pi}{2}$ anticlockwise takes us from $1$ to $i$. Then we place $\Sigma$ centred at the origin in $\mathbb{C}$ so that the intersection between $\Sigma$ and $\mathbb{C}$ is the unit circle in $\mathbb{C}$ which runs along the equator of the sphere. A stereographic projection, defined in the following slides, will be used to assign points of $\mathbb{C}_{\infty}$ to the sphere $\Sigma$. This stereographic image is a one-to-one correspondence between points in $\mathbb{C}$ and $\Sigma$. 
\end{frame} 

\begin{frame}{Stereographic Projection}
\begin{remark}
The south pole of $\Sigma$ is the point $(0,0,-1)$ and the north pole of $\Sigma$ is the point $(0,0,1)$ if we described the setup in Cartesian coordinates.
\end{remark}

\subsubsection{Some background}
The stereographic projection method was first used by Ptolemy to map positions of astronomical objects on the celestial sphere.  This gave Ptolemy a way to map objects in the sky to points on the sphere, as they are seen by an observer on Earth.
\end{frame}

\begin{frame}{Correspondence between points of $\Sigma$ and $\mathbb{C}_{\infty}$}
To begin with, take a line from the north pole $N$ of $\Sigma$ to any point $p$ in $\mathbb{C}$. 
\begin{definition}
The point $\hat{p}$ on $\Sigma$ where the line from $N$ to $p$ intersects $\Sigma$ is called the stereographic image of $p$ on $\Sigma$. 
\end{definition}

\begin{remark}
Notice that for any point $p$ in the complex plane as $p$ moves away from the origin in any direction (equivalently for $z\in\mathbb{C}$ we have $|z|\to\infty$) then $\hat{p}$ the stereographic image of $p$ on $\Sigma$ gets closer and closer to $N$ as $p$ gets further away from the origin. From this we define that the stereographic image on $\Sigma$ of $\infty\in\mathbb{C}_{\infty}$ is $N$.
\end{remark}
\end{frame}

\begin{frame}{Correspondence between points of $\Sigma$ and $\mathbb{C}_{\infty}$}
We list some facts arising from the stereographic projection:
\begin{enumerate}
    \item The stereographic image on $\Sigma$ of any point $p$ of $\mathbb{C}$ which lies inside the unit circle is on the southern hemisphere of $\Sigma$.
    \item The stereographic image of 0 on $\Sigma$ is the south pole of $\Sigma$.
   % \item The stereographic image on $\Sigma$ of any point $p$ on the unit circle is the point $p$ itself. This is because the unit circle coincides with the equator of $\Sigma$ and so the line from $N$ to $p$ intersects $\Sigma$ and $\mathbb{C}$ at the same point, namely $p$.
    \item The stereographic image on $\Sigma$ of any point $p$ in $\mathbb{C}$ which lies outside of the unit circle is on the northern hemisphere of $\Sigma$.
    %\item The point $N$ on $\Sigma$ is not the stereographic image of any point in $\mathbb{C}$.
\end{enumerate}
\begin{definition}
Now that we have used stereographic projection to assign points of $\mathbb{C}_{\infty}$ to points of $\Sigma$, we call $\Sigma$ the Riemann sphere.
\end{definition}
\end{frame}


\begin{frame}{A Visual representation}
\begin{figure}
    \centering
     \includegraphics[scale=0.75]{images/sphere.jpg}
    
    \caption{Riemann Sphere}
    \label{fig:my_label}
\end{figure}
\end{frame}


\end{document}