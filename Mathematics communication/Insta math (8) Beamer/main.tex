% Inbuilt themes in beamer
\documentclass{beamer}

% Theme choice:
\usetheme{Copenhagen}
\usecolortheme{wolverine}
\usepackage{graphicx}
\usepackage{wrapfig}
\usepackage[absolute,overlay]{textpos}
\usepackage{pgfplots}
\graphicspath{ {./images/} }


% Title page details: 
\title{Topological Dynamics} 

\author{}
\date{}
\logo{}


\newtheorem*{remark}{Remark}
\newtheorem{prop}{Proposition}[section]
\newtheorem*{notation}{Notation}
\newtheorem*{ex}{Examples:}



\begin{document}

% Title page frame
\begin{frame}
    \titlepage 
    
    
\begin{textblock*}{10cm}(1cm,4.5cm)
This investigates:
\begin{enumerate}
        \item Time Discrete Dynamical Systems
        \item Examples
        \item Fixed Points
        \item Periodic Points
        \item Orbits
    \end{enumerate}
\end{textblock*}    
\end{frame}

% Remove logo from the next slides
\logo{}


% Outline frame
\begin{frame}{Introduction}
\begin{definition}
Let $X$ be a topological space and $f:X\to X$ be continuous. Then we call the pair $(X,f)$ a \underline{topological dynamical system}. 
\end{definition}

\begin{remark}
    We can think of the map $f^{n}=f\circ f\circ \dots \circ f$ as moving points in our space $X$ forwards $n$ steps in time.
    We have $f^{0}=id_{X}$ and $f^{m+n}=f^{m}\circ f^{n}$.
\end{remark}

\begin{remark}
    Note that $n\in\mathbb{N}$ as we are considering a time-discrete system.
    If we take $t\in[0,\infty)$ and consider $f^{t}$ then this would give us a time continuous system.
\end{remark}

\end{frame}


% Lists frame

\begin{frame}{Examples}
\begin{example}[Tent Map]
For any $k\in[0,2]$ define the tent map with slope $k$, $T_{k}:[0,1]\to[0,1]$ as follows,
\begin{equation*}
x\mapsto\frac{k}{2}(1-\vert2x-1\vert)=
\begin{cases}
    & kx, \text{if } x\in[0,\frac{1}{2}], \\
    & k(1-x) \text{if } x\in[\frac{1}{2},1]\\
\end{cases}
\end{equation*}

\end{example}
We show a graph of $T_{2}$, called the full tent map, on the next slide.
\end{frame}
    
\begin{frame}{Examples}
\begin{tikzpicture}
\begin{axis}[
title={\underline{Full Tent Map}},
    axis lines = left,
    xlabel = \(x\),
    ylabel = {\(y\)},
]
%Below the red parabola is defined
\addplot [
    domain=0:0.5, 
    samples=100, 
    color=blue,
]
{2*x};
\addlegendentry{$y=T_{2}(x)$}
%Here the blue parabola is defined
\addplot [
    domain=0.5:1, 
    samples=100, 
    color=blue,
    ]
    {2*(1-x)};


\end{axis}

\end{tikzpicture}

\end{frame} 

\begin{frame}{Examples}
\begin{example}[Logistic Map]
Let $\mu>0$ and define $\Lambda_{\mu}:\mathbb{R}\to\mathbb{R}$, by $x\mapsto\mu x(1-x)$.
\end{example}


\begin{tikzpicture}
\begin{axis}[height=7.5cm,width=11cm][
title={\underline{Logistic Map}},
    axis lines = left,
    xlabel = \(x\),
    ylabel = {\(y\)},
    axis labels at tip
]
%Below the red parabola is defined


%Here the blue parabola is defined
\addplot [
    domain=0:1, 
    samples=100, 
    color=blue,
    ]
    {1*x*(1-x)};
\addlegendentry{$y=\Lambda_{\mu}(x),\mu=1$}

\end{axis}

\end{tikzpicture}


\end{frame}

\begin{frame}{Examples of Contractions}

\end{frame}

\begin{frame}{Contraction Mapping Theorem}



\end{frame}

\begin{frame}{Contraction Mapping Theorem}


    
\end{frame}

\begin{frame}{An Application}

\end{frame}


\end{document}

