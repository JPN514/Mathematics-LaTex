% Inbuilt themes in beamer
\documentclass{beamer}

% Theme choice:
\usetheme{Copenhagen}
\usepackage{graphicx}
\usepackage{wrapfig}
\usepackage[absolute,overlay]{textpos}
\graphicspath{ {./images/} }

% Title page details: 
\title{Outer Lebesgue Measure} 

\author{}
\date{}
\logo{\large \LaTeX{}}


\newtheorem*{remark}{Remark}
\newtheorem{prop}{Proposition}[section]
\newtheorem*{notation}{Notation}
\newtheorem*{ex}{Examples:}


\begin{document}

% Title page frame
\begin{frame}
    \titlepage 
    
    \begin{wrapfigure}{r}{0.25\textwidth}
    \centering
    \includegraphics[width=0.25\textwidth]{images/Lebesgue_2.jpeg}
    
    
\end{wrapfigure}
\begin{textblock*}{10cm}(1cm,4.5cm)
This investigates:
\begin{enumerate}
        \item Outer Lebesgue Measure
        \item Properties
        \item Examples 
    \end{enumerate}
\end{textblock*}    
\end{frame}

% Remove logo from the next slides
\logo{}


% Outline frame
\begin{frame}{Introduction}
\begin{definition}[Outer Lebesgue Measure]
Let $E$ be a subset of $\mathbb{R}$. The Outer Lebesgue Measure $\lambda^{*}(E)$ is the non-negative number defined as 
\begin{align*}
\lambda^{*}(E)=\text{inf}\big\{&\sum_{j=1}^{\infty}\vert I_{j}\vert:\text{every } I_{j} \text{ is an open or} \\ & \text{empty interval and } \bigcup_{j=1}^{\infty}I_{j}\supseteq E\big\}
\end{align*}
where inf is the infimum, which we can think about as the greatest lower bound of the set $E$.
\end{definition}
\end{frame}


% Lists frame

\begin{frame}{Remarks on $\lambda^{*}$}
\begin{block}{Definition (continued).}
\begin{enumerate}
    \item For every open interval $I_{j}=(a_{j},b_{j})$, we have $a_{j}\leq b_{j}$, and we set $\vert I_{j}\vert = b_{j}-a_{j}$;
    \item Allowing $a_{j}=b_{j}$, we have $I_{j}=\emptyset$, so $\vert I_{j}\vert=0$.
\end{enumerate}
\end{block}

\begin{remark}
\begin{enumerate}
    \item $\lambda^{*}(E)$ is defined for all $E\subseteq\mathbb{R}$ and we always have $\lambda^{*}(E)\geq 0$;
    \item $\lambda^{*}(E)$ can be equal to $+\infty$;
    \item For $I_{j}$'s open intervals with $\bigcup_{j\geq 1}I_{j}\supseteq E$ then we have \[\lambda^{*}(E)\leq\sum_{j=1}^{\infty}\vert I_{j}\vert\text{.}\tag{$1$}\label{eqn:(1)}\]
\end{enumerate}
\end{remark}

\end{frame}
    
\begin{frame}{Properties of Outer Lebesgue Measure}
\begin{block}{Basic Properties}
\begin{enumerate}
    \item $\lambda^{*}(\emptyset)=0$.
    \item For all $x\in\mathbb{R}$, we have $\lambda^{*}(\{x\})=0$.
    \item For $E_{1},E_{2}\subseteq\mathbb{R}$ with $E_{1}\subseteq E_{2}$, then $\lambda^{*}(E_{1})\leq \lambda^{*}(E_{2})$.
    \item For a countable collection of sets $\big(E_{k})_{k\geq1}$ we have, $$\lambda^{*}\big(\bigcup_{k=1}^{\infty}E_{k}\big)\leq \sum_{k=1}^{\infty}\lambda^{*}(E_{k}).$$
    \item For all $x\in\mathbb{R}$ and any $E\subseteq\mathbb{R}$, $\lambda^{*}(E+x)=\lambda^{*}(E)$, where $E+x=\{y+x:y\in E\}$ is the translation of $E$.
\end{enumerate}
\end{block}
\end{frame} 

\begin{frame}{Finite and Countable Sets with Examples}
\begin{example}
For any $E\subseteq\mathbb{R}$ which is finite or countable, we have $\lambda^{*}(E)=0$. In particular, $\lambda^{*}(\mathbb{Q})=0$.
\end{example}
\begin{proof}
Consider $E=\{q_{1},q_{2},...\}$ with $q_{i}\in\mathbb{R}$ for all $i$. We know $0\leq\lambda^{*}(E)$ and that $E$ is the union of a countable collection of sets $\{q_{i}\}$, so $E=\bigcup_{i=1}^{\infty}\{q_{i}\}$. We then have,
$$\lambda^{*}(E)=\lambda^{*}\big(\bigcup_{i=1}^{\infty}\{q_{i}\}\big)
\leq\sum_{i=1}^{\infty}\lambda^{*}(\{q_{i}\})=\sum_{i=1}^{\infty}0=0,$$
as $\lambda^{*}(\{q_{i}\})=0$ for all $i$. Thus, $\lambda^{*}(\mathbb{Q})=0$.
\end{proof}
\end{frame}

\begin{frame}{Further Examples}
\begin{Examples}
\begin{enumerate}
    \item For a closed interval $I=[a,b]$, the outer Lebesgue measure coincides with its length, that is, $\lambda^{*}(I)=b-a$.
    \item $\lambda^{*}(\mathbb{R})=+\infty$.
    \item $\lambda^{*}([0,1]\setminus\mathbb{Q})=1.$
\end{enumerate}
\end{Examples}

\end{frame}

\begin{frame}{Further Examples}
\begin{proof}[Proof of 3]
Before we begin note that, $\lambda^{*}([0,1])=1$ as $[0,1]$ is a closed interval. Furthermore, we have that $\mathbb{Q}\cap[0,1]$ is countable since it is simply all the rational numbers within $[0,1]$ and that $([0,1]\setminus\mathbb{Q})\subseteq[0,1]$. We can then use basic properties $3$ and $4$ of the outer Lebesgue measure to show
\begin{align*}
 1= \lambda^{*}([0,1])&=\lambda^{*}\big((\mathbb{Q}\cap[0,1])\cup([0,1]\setminus\mathbb{Q})\big)
 \\& \leq \lambda^{*}\big((\mathbb{Q}\cap[0,1])\big) + \lambda^{*}\big(([0,1]\setminus\mathbb{Q})\big) 
 \\& = 0 + \lambda^{*}\big(([0,1]\setminus\mathbb{Q})\big)
 \\& \leq \lambda^{*}([0,1]) = 1.
\end{align*}
Thus, $\lambda^{*}\big([0,1]\setminus\mathbb{Q}\big)=1$.
\end{proof}
\end{frame}



\end{document}
